\documentclass[11pt,legalpaper]{article}
\usepackage[utf8]{inputenc}
\usepackage[T1]{fontenc}
\usepackage[spanish]{babel}


\author{Alexander Gallardo}
\title{Apuntes de \LaTeX{} 2}
\date{}

\begin{document}
	\maketitle
	\begin{flushleft}%Alineación a la izquierda
		\section{Ejercicio} %Seccion 1
			Para obtener una simple tabla sin lineas:
			\begin{center}%Centra la tabla
				\begin{tabular}{l c r}
					1 & 2 & 3\\
					4 & 5 & 6\\
					7 & 8 & 9\\	
				\end{tabular}
			\end{center}
		\section{Ejercicio}
		 Para obtener esta tabla en la que añadimos algunas lineas verticales:\\
			\begin{center}
				\begin{tabular}{l| c|| r|}
					1 & 2 & 3\\
					4 & 5 & 6\\
					7 & 8 & 9\\	
				\end{tabular}
			\end{center}
		\section{Ejercicio}
		Ahora con lineas horizontales: superior e inferior\\
			\begin{center}
				\begin{tabular}{l| c|| r|}
					\hline
						1 & 2 & 3\\
						4 & 5 & 6\\
						7 & 8 & 9\\
					\hline	
				\end{tabular}
			\end{center}
		\section{Ejercicio}
		Ahora añadir lineas centradas entre todas las filas (usamos el entorno \textbf{center})
			\begin{center}
				\begin{tabular}{l| c|| r|}
					\hline
					10 & 22 & 33\\ \hline
					44 & 55 & 66\\ \hline
					77 & 88 & 99\\
					\hline	
				\end{tabular}
			\end{center}
	\end{flushleft}
\end{document}