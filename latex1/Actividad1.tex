%-------------------------Preambulo----------------------
\documentclass[12pt, a4papper]{article} %Crea el documento y define los parametros
\usepackage[utf8]{inputenc} %Codificacion de caracteres
\usepackage[spanish]{babel} %Define el Idioma
%\textbf{} es para poner en negrita
\author{Alexander Gallardo}
\title{TALLER DE OPERACION}
\date{}

%-------------------------Cuerpo------------------------
\begin{document}
	\maketitle
	
	\begin{center}%Centrado
		\textbf{Apuntes de \LaTeX {} 1}
	\end{center}

	\begin{flushleft}
		¡Hola mundo!Esto de trabajar con \LaTeX{} es maravilloso, es el mejor editor de textos.\\
		Con \LaTeX{} puede hacer muchas más cosas que con otros editores.
		¿Alguna vez habías imaginado que podías tener una imprenta en tus manos?
	\end{flushleft}

	\begin{center}
		{\sffamily \bfseries INTRODUCCIóN A}\\
		{\Huge \LaTeX} 
	\end{center}
	
	\begin{flushleft}
		En este curso aprenderas lo minimo, que necesitas saber para poder crear documentos cientificios de calidad con \LaTeX.
		
		Recuerda que \LaTeX {} es un paquete basado en \TeX, que es el procesador de textos que siempre se emplea.
	\end{flushleft}

\end{document}